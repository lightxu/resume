% !TEX TS-program = xelatex
% !TEX encoding = UTF-8 Unicode
% !Mode:: "TeX:UTF-8"

\documentclass{resume}
\usepackage{zh_CN-Adobefonts_external} % Simplified Chinese Support using external fonts (./fonts/zh_CN-Adobe/)
%\usepackage{zh_CN-Adobefonts_internal} % Simplified Chinese Support using system fonts
\usepackage{linespacing_fix} % disable extra space before next section
\usepackage{cite}

\begin{document}
\pagenumbering{gobble} % suppress displaying page number

\name{Shicheng Xu}

% {E-mail}{mobilephone}{homepage}
% be careful of _ in emaill address
\contactInfo{lightxuzju@gmail.com}{(+1) 412-980-5297}{http://lightxu.com}
% {E-mail}{mobilephone}
% keep the last empty braces!
%\contactInfo{xxx@yuanbin.me}{(+86) 131-221-87xxx}{}
 
\section{\faGraduationCap\ Education}
\datedsubsection{\textbf{Carnegie Mellon University, Silicon Valley}, Mountain View, CA}{Jan. 2016 -- Present}
\textit{Master student} in Software Engineering, expected June 2017
\datedsubsection{\textbf{Zhejiang University}, Hangzhou, China}{Sept. 2010 -- Oct. 2014}
\textit{B.S.} of Engineering in Computer Science\\
Entered Pursuit Science Class (Computer Science) at Chu Kochen Honors College

\section{\faUsers\ Experience}
\datedsubsection{\textbf{SSNoC}, Course Team Project}{Jan 2016 -- Present}
\role{Full-stack developer}{Foundation of Software Engineering}
In a natural disaster scenario, Survivable Social Network on a Chip(SSNoC) can provide a local network that allows
users to touch base with their neighbors and exchange information.
\begin{itemize}
  \item Develop a web application on a beaglebone chip with Sqlite/ExpressJS/AngularJS/NodeJS
  \item Learn front-end/backend testing with Selenium and Mocha
  \item Give a presentation on ``Enhancing front-end development with AngularJS and jQuery''
\end{itemize}

\datedsubsection{\textbf{Twitter Analytics Web Service}, Course Team Project}{Jan 2016 -- Present}
\role{Full-stack developer}{Cloud Computing}
Build a performant and reliable web service on the cloud within a specified budget that can process and analyze massive twitter data.
\begin{itemize}
  \item Design, develop, deploy and optimize functional web-servers that can handle a high load (\~{}tens of thousands of requests per second).
  \item Implement Extract Transform and Load (ETL) on a large data set (\~{}1 TB) and load into MySQL and HBase systems.
\end{itemize}

\datedsubsection{\textbf{Carnegie Mellon University}, Pittsburgh, PA}{July 2014 -- Dec. 2015}
\role{Visiting Researcher}{Professor: Alex Hauptmann}
Aladdin Video Project is short for Automated Low-Level Analysis and Description of Diverse INtelligence\\
E-LAMP Prototype System is a content-based video event search and content tag creation system
\begin{itemize}
  \item Standardizing preprocessing modules by defining Dockerfiles and building Docker images
  \item Implementing a job scheduler to run modules on AWS with ECS, Dynamo DB, and S3
  \item Implemented a user interface to visualize a backend machine learning model, which helps the
user to incrementally refine a video query
  \item Developed a web application to connect the user interface to backend search modules
\end{itemize}

\datedsubsection{\textbf{NIST TRECVid MER 2014}}{July 2014 -- Sept. 2014}
\role{Participant}{CMU Informedia Group}
NIST TRECVid is a contest on video retrieval evaluation to encourage research in automatic segmentation, indexing, and content-based retrieval of digital video\\
Multimedia event recounting(MER) recounts the evidence for the presence of the detected event in each search video determined to contain it
\begin{itemize}
  \item Analyzed and defined the taxonomy on 2500+ concept detectors for semantic query searches
  \item Built a pipeline to recount event detection results from event detectors and semantic queries
  \item The system utilizes shortest video duration and get 57\% positive response
\end{itemize}

\datedsubsection{\textbf{NIST TRECVid MED 2013}}{Fall 2013}
\role{Participant}{CMU Informedia Group}
NIST TRECVid is a contest on video retrieval evaluation to encourage research in automatic segmentation, indexing, and content-based retrieval of digital video\\
The goal of MED(Multimedia Event Detection) is to assemble core detection technologies into a system that can search multimedia recordings for user-defined events based on pre-computed metadata
\begin{itemize}
  \item Responsible for feature extraction(SIFT/Color SIFT/MFCC/Dense Trajectory)
  \item  Implemented Cascade SVM on Hadoop for paralyzing detector training 
\end{itemize}

% Reference Test
%\datedsubsection{\textbf{Paper Title\cite{zaharia2012resilient}}}{May. 2015}
%An xxx optimized for xxx\cite{verma2015large}
%\begin{itemize}
%  \item main contribution
%\end{itemize}

\section{\faCogs\ Skills}
\begin{itemize}[parsep=0.5ex]
  \item Programming Languages: C == (Python, Ruby, Javascript) > (C++, Java, PHP) > (Bash, Awk, MATLAB)
  \item Platform: (Debian, Ubuntu, Fedora) == Mac > Windows
  \item Cloud service: AWS > Rocks Cluster > Hadoop > Azure
\end{itemize}

\section{\faHeartO\ Honors and Awards}
\datedline{\textbf{Best demo award} for E-LAMP system at International Conference on Multimedia Retrieval }{2015}
\datedline{\textbf{Best performer} on Multimedia Event Detection task at NIST TRECVid }{2014}
\datedline{\textbf{Outstanding Graduates} of Zhejiang University }{2014}
\datedline{\textbf{Excellent Student Awards} of Zhejiang University }{2011, 2012, 2013}
\datedline{\textbf{Silver Prize} on The 9th Zhejiang Provincial ACM/ICPC Collegiate Programming Contest }{2012}
\datedline{\textbf{Scholarship for Excellence in Research and Innovation} of Zhejiang University }{2011}
\datedline{\textbf{First Prize}, National Olympiad in Informatics in Provinces (NOIP) }{Jun. 2013}

\section{\faInfo\ Miscellaneous}
\begin{itemize}[parsep=0.5ex]
  \item Languages: English - Professional, Mandarin, Wu - Native speaker, French - Beginner
  \item Document: \LaTeX = MarkDown > (Haml, Jade, Jinja2) > HTML
  \item Instruments: Electrical Guitar, Electrical Piano
  \item Sports: Swimming, Biking, Running, Tennis, Table tennis
  \item GitHub: https://github.com/lightxu
  \item Twitter: https://twitter.com/shichengxu
  \item Facebook: https://www.facebook.com/lightxuzju
\end{itemize}

\section{\faUsers\ Activities}
\datedline{\textbf{Budai(布袋)} marathon running team of Zhejiang University }{2012 – 2014}
\datedline{\textbf{Full marathon (5h2m30s)}, Dick's Sporting Goods Pittsburgh Marathon }{2015}
\datedline{\textbf{Half marathon}, ASICS Hangzhou Mountain Cross Marathon }{2013}
\datedline{\textbf{Full marathon (4h5m45s)}, C\&D Xiamen International Marathon }{2013}
\datedline{\textbf{Full marathon (3h52m45s)}, Hangzhou International Marathon }{2012}
\datedline{\textbf{Bronze medal}, Zhejiang University Sports Meeting Men's 5000-meters Race }{2012}
\datedline{Winner of monthly champion in the CCTV Samsung Intelligent Express Show}{2007}

%% Reference
%\newpage
%\bibliographystyle{IEEETran}
%\bibliography{mycite}
\end{document}
